%        File: Mathematiques.effectives.tex
%     Created: Don Okt 15 06:00  2015 C
% Last Change: Don Okt 15 06:00  2015 C
%
\documentclass[a4paper, 11pt]{report}

\usepackage[french]{babel}
\usepackage[T1]{fontenc}
\usepackage[utf8]{inputenc}
\usepackage{hyperref}

\usepackage{amsmath}
\usepackage{amsthm}
\usepackage{amssymb}
\usepackage{physics}

\usepackage{amsmath}

%-------------------------------------------------------------------------------
\newcommand{\naturel}{\mathbb{N}}
\newcommand{\integer}{\mathbb{Z}}
\newcommand{\rational}{\mathbb{Q}}
\renewcommand{\real}{\mathbb{R}} %Already defined in physics package
\newcommand{\complex}{\mathbb{C}}
%-------------------------------------------------------------------------------

%Logical
\def\implies{\Rightarrow}
\def\equiv{\Leftrightarrow}
%-------------------------------------------------------------------------------

% Calculability
\def\recursiveFunctionsSet{\mathcal{R}}
\def\primitiveRecursiveFunctionSet{\mathcal{RP}}
\def\muRecursiveFunctionSet{\mathcal{R}_{\mu}}
%-------------------------------------------------------------------------------

%Set theory
\def\union{\cup} %Union
\def\inter{\cap} %Intersection
\newcommand{\comp}[1]{#1^{c}} %Complementary
\def\cartprod{\cross}
\newcommand{\cardinal}[1]{|#1|}
\newcommand{\sequencesIn}[1]{#1^{\omega}}
%-------------------------------------------------------------------------------

%Topology
\def\interior{\mathring}
\def\adh{\overline}
%-------------------------------------------------------------------------------

%Algebra

% Linear algebra

\newcommand{\matrixSpace}[2]{M_{#1}(#2)}
\newcommand{\inversibleMatrixSpace}[2]{GL_{#1}(#2)}
\newcommand{\spanspace}[1]{\left<#1\right>}

% Group theory

\def\isomorphe{\simeq}
\newcommand{\ordergroup}[1]{|#1|}
\newcommand{\GSautomorphismDef}[2]{Aut_{\text{#1}}(#2)}
\newcommand{\generatedGroup}[1]{\left<#1\right>}
\newcommand{\permutationGroup}[1]{\mathfrak{S}(#1)}

% Field theory

%extensionDegree
\newcommand{\extensionDegree}[2]{[#1 : #2]}

\newcommand{\extensionField}[2]{#1/#2}

%plongement
\newcommand{\plongement}[3]{Hom_{#1}(#2, #3)}

%Galois group
\newcommand{\galoisGroup}[2]{G(#1, #2)}

% Spectrum Theory
\newcommand{\spectrum}[1]{\sigma(#1)}
\newcommand{\resolvant}[1]{\rho(#1)}

\def\Ldeux{\mathcal{L}^{2}}
\def\Ldeuxstar{(\mathcal{L}^{2})^{*}}

%GSsequence :
%		#1 : represention of elements of the sequences
%		#2 : indices
%		#3 : set definition
\newcommand{\GSsequence}[3]{(#1_{#2})_{#2 \in #3}}

%GSsetDef :
%		#1 : set elements
\newcommand{\GSset}[1]{\left\{ #1 \right\}}

%GSsetDef :
%		#1 : global set
%		#2 : condition
\newcommand{\GSsetDef}[2]{\left\{#1 \, | \, #2 \right\}}

%GSprodSet :
%		#1 : indice
%		#2 : begin indice
%		#3 : end indice
%		#4 : set
\newcommand{\GSprodSet}[4]{\displaystyle \prod_{#1 = #2}^{#3} #4_{#1}}

%GSsum :
%		#1 : indice
%		#2 : begin indice
%		#3 : end indice
%		#4 : element
\newcommand{\GSsum}[4]{\displaystyle \sum_{#1 = #2}^{#3} #4}

\newcommand{\GSintervalCC}[2]{\left[#1, #2\right]}
%-------------------------------------------------------------------------------

%Analysis :

% conjuguate
\def\conjuguate{\overline}

% miLength: multi-indice length
\newcommand{\miLength}[1]{|#1|}

% segment: all points between two points given
\newcommand{\segment}[2]{S(#1, #2)}

%GSfunction :
%       #1 : name function
%       #2 : begin set
%       #3 : end set
\newcommand{\GSfunction}[3]{#1 : #2 \rightarrow #3}

%GSnorme : Deprecated --> \norm
%		#1 : elements which norme is applied on
\newcommand{\GSnorme}[1]{\norm{#1}}

%GSnormeDef :
%		#1 : elements which norme is applied on
%		#2 : norme indice
\newcommand{\GSnormeDef}[2]{\norm{#1}_{#2}}

%GSnormedSpace :
%		#1 : vectorial space
%		#2 : \GSnorme[Def] with dot as element.
\newcommand{\GSnormedSpace}[2]{(#1, #2)}

%Identification
\def\identification{\simeq}

%GSdual
%		#1 : vectorial space
\newcommand{\GSdual}[1]{#1^{*}}

%GSbidual
%		#1 : vectorial space
\newcommand{\GSbidual}[1]{#1^{**}}

\newcommand{\GSunitBoule}[1]{\mathcal{B}_{#1}}
\newcommand{\GSclosedUnitBoule}[1]{\adh{\GSunitBoule{#1}}}

\newcommand{\GSweakTopo}[1]{\sigma(#1, #1^{*})}
\newcommand{\GSpreweakTopo}[1]{\sigma(#1^{*}, #1)}

%GSendomorphism
\newcommand{\GSendomorphism}[1]{End(#1)}

%GShomomorphisme
\newcommand{\GShomomorphisme}[3][]
{
	Hom_{#1}(#2, #3)
}

%GShomomorphismeDef
% Deprecated !! Use instead directly \GShomomorsphisme
\newcommand{\GShomomorphismeDef}[3][]
{
	\GShomomorphisme{#1}{#2}{#3}
}

%GScontinueEndo
\newcommand{\GScontinueEndo}[2][]
{
	\mathcal{L}_{#1}(#2)
}

%\GScontinueHomo
\newcommand{\GScontinueHomo}[3][]
{
	\mathcal{L}_{#1}(#2; #3)
}

%\GScompactEndo
\newcommand{\GScompactEndo}[1]{\mathcal{K}(#1)}

%\GScompactHomo
\newcommand{\GScompactHomo}[2]{\mathcal{K}(#1; #2)}

\newcommand{\GSfiniteRankHomo}[2]{\mathcal{R}_{f}(#1; #2)}
\newcommand{\GSfiniteRankEndo}[1]{\mathcal{R}_{f}(#1)}

\newcommand{\GSisomorphisme}[1]{Isom(#1)}
\newcommand{\GSisomorphismeHomo}[2]{Isom(#1; #2)}
\newcommand{\GSisometryEndo}[1]{Isom(#1)}

\newcommand{\jacobienneMatrix}[2]{J_{#1}(#2)}
\newcommand{\hessienneMatrix}[2]{\mathcal{H}_{#1}(#2)}

\newcommand{\hyperreal}{ { }^{*} \real}
%-------------------------------------------------------------------------------
%Model theory

\def\satisfies{\vdash}
\newcommand{\lang}[1]{\mathcal{#1}}
\newcommand{\theory}[1]{\mathcal{#1}}
\newcommand{\struct}[1]{\mathcal{#1}}

%Definissable set
%	1 : order of the definissable sets
%	2 : the l-structure which we define the definissable sets on.
\newcommand{\definissableSet}[2][]
{
	Def^{#1}(#2)
}

%Type set
%	1 : n if it's the set of n-types.
%	2 : the theory which we build the types on.
\newcommand{\typeSet}[2]{S_{#1}(#2)}

%Ultraproduct
%	1 : indice elements
%	2 : set which contains indices
%	3 : ultrafilter
%	4 : models represention
\newcommand{\GSultraproduct}[4]{\displaystyle {\prod_{#1 \in #2}}^{#3}#4_{#1}}

%Ultrapower
%	1 : indice elements
%	2 : set which contains indices
%	3 : ultrafilter
%	4 : model
\newcommand{\ultrapower}[4]{\displaystyle {\prod_{#1 \in #2}}^{#3} #4}

%Substructures
\def\substructure{\subseteq}

%Elementary Substructures.
\def\elemSubstructure{\preceq}

\def\existentiallyClosed{\underset{e.c}{\subseteq}}
%-------------------------------------------------------------------------------

%	Hilbert space
\def\Hilbert{\mathcal{H}}
\newcommand{\GSortho}[1]{#1^{\perp}}
\def\GSid{\cong}
\newcommand{\dotprod}[2]{\bra{#1}\ket{#2}}
\newcommand{\adjointe}[1]{#1^{*}}
%-------------------------------------------------------------------------------

%	Group representions
\newcommand{\GSrepr}[2]{Repr(#1, #2)}
\newcommand{\GSreprf}[2]{Repr_{f}(#1, #2)}
\newcommand{\GSrepri}[2]{Repr_{i}(#1, #2)}
%-------------------------------------------------------------------------------

%Probability

%	Borelian
\newcommand{\borelian}[1]{\mathcal{B}(#1)}

%	Laws
\newcommand{\lawBernouilli}[1]{\mathcal{B}(1, #1)}
\newcommand{\lawBinomial}[2]{\mathcal{B}(#1, #2)}
\newcommand{\lawExponential}[1]{e(#1)}
\newcommand{\lawPoisson}[1]{\mathcal{P}(#1)}
\newcommand{\lawNormal}[2]{\mathcal{N}(#1, #2)}
\newcommand{\lawUniform}[1]{\mathcal{U}(#1)}
\newcommand{\lawCauchy}{\mathcal{C}}

%	Variables
\def\varFollow{\sim}
\def\varSameLaw{\overset{\mathcal{L}}{=}}
\def\varIndependant{\perp}

%-------------------------------------------------------------------------------

\usepackage{amsfonts}
\usepackage{amssymb}
\usepackage{amsmath}
\usepackage{amsthm}
\usepackage{mathrsfs}

\newtheorem{definition}{Définition}[chapter]

\newtheorem{proposition}[definition]{Proposition}
\newtheorem{lemma}[definition]{Lemme}
\newtheorem{corollary}[definition]{Corollaire}
\newtheorem{theorem}[definition]{Théorème}

\newtheorem*{exemple}{Exemple}
\newtheorem*{question}{Questions}
\newtheorem*{remarque}{Remarque}
\newtheorem*{notation}{Notation}

\newtheorem{exercice}{Exercice}[chapter]


\title{Mathématiques effectives}
\author{Danny Willems}
\date{Version compilée du \today}

\begin{document}

\maketitle

\tableofcontents

\chapter*{Introduction}

Le cours de mathématiques effectives se charge d'étudier la théorie des jeux.

% Définition d'un jeu (avec du vocabulaire français
% Expliquer notre démarche d'étude
% Donner la définition d'un jeu à information parfaite et à somme nulle.

\chapter{Jeux combinatoires/qualitatifs à information parfaite et à somme nulle}

\section{Exemples de jeux}

Commençons par donner deux exemples de jeux.

\subsection*{Jeu de Nim 1}

Le jeu de Nim est un jeu qui se joue à deux, chacun à son tour. Il est composé
de deux tas d'objets. Il y a $n_{1}$ objets dans le tas $1$ et $n_{2}$ objets
dans le tas $2$. On se retrouve donc avec un couple $(n_{1}, n_{2}) \in
\naturel^{2}$.

Les règles sont simples: le jeu se joue tour à tour. A chaque tour, le joueur
courant doit retirer $k$ objets dans un des deux tas. Les $k$ pièces doivent
être retirées dans un même tas.

Un joueur gagne lorsque celui-ci a retiré le ou les derniers objets, l'autre
joueur a alors perdu.

\begin{remarque}
	Le jeu de Nim 1 a pour caractéristiques:
	\begin{enumerate}
		\item Il se joue à deux joueurs. Notons les $J_{1}$ et $J_{2}$.
		\item L'état du jeu est connu en entier par les deux joueurs à chaque
			coup. Le jeu est donc à information parfaite.
		\item Le jeu se joue tour à tour.
		\item $J_{1}$ gangne ssi $J_{2}$ perd.
		\item Aucun match nul possible.
	\end{enumerate}
\end{remarque}

\subsection*{Jeu de Nim 2}

\section{Définitions}

Nous allons tenter de donner une définition de jeu englobant les jeux de Nim
cités précédemment. On créera ainsi une classe de jeu, et on tentera de retirer
des théorèmes intéressants. Ces théorèmes pourraient nous donner des conditions
nécessaires et suffisantes pour gagner à un jeu de cette classe, d'exitence de
telles conditions, nous la donner, etc. Est-ce possible déjà ?

Pour classer ce jeu, on va utiliser la notion de graphe.

\begin{definition} [Graphe]
	\textbf{Un graphe} ou \textbf{un espace de graphe} est un couple $(V, E)$ où
	$V$ est un ensemble et $E \subseteq V \cartprod V$.

	Si $V$ est fini, on parle alors \textbf{de graphe fini} ou encore
	\textbf{d'espace de graphe fini}. Si $V$ n'est pas fini, on parle
	\textbf{de graphe infini} ou encore \textbf{d'espace de graphe infini}.

	Les éléments de $V$ sont appelés \textbf{les sommets} et les éléments de $E$
	sont appelés \textbf{les arcs}.
\end{definition}

\begin{definition} [Graphe non orienté]
	Soit $(V, E$) un graphe.

	On dit que $(V, E)$ est \textbf{un graphe orienté} si pour tous sommets
	$v_{1}, v_{2}$, on a
	\begin{equation}
		(v_{1}, v_{2}) \in E \equiv (v_{2}, v_{1}) \in E
	\end{equation}
	Si ce n'est pas le cas, on parle de \textbf{graphe non orienté}.
\end{definition}

\begin{definition} [Jeu combinatoire]
	\textbf{Un jeu combinatoire} $G$ est un 5-uplet $( (V, E), V_{1}, V_{2},
	\Omega_{1}, \Omega_{2})$ tel que

	\begin{enumerate}
		\item $(V, E)$ est un graphe fini.
		\item $\GSset{V_{1}, V_{2}}$ forment une partition de $V$.
		\item $\Omega_{1} \subseteq \sequencesIn{V}$ et $\Omega_{2} \subseteq
			\sequencesIn{V}$.
		\item Pour tout $v \in V$, il existe $v' \in V$ tel que $(v, v') \in E$.
	\end{enumerate}
\end{definition}

Remarquons qu'actuellement, un jeu combinatoire est défini de manière purement
syntaxique. Il nous reste à donner l'interprétation de ce $5$-uplet.

Dans le jeu de Nim décrit dans l'introduction, on a remarqué que le jeu était
défini par des états. Ces états seront définis par les sommets du graphe.

Les actions possibles des joueurs seront les arcs de ce graphe. Remarquons que
l'action 'revenir' en arrière n'est pas toujours possible à cause de
l'utilisation d'un graphe orienté.

$V_{1}$ (resp. $V_{2}$) représente les états accessibles par le joueur $1$
(resp. par le joueur $2$). Les états accessibles à partir d'un état $v$ sont
donnés par les arcs ayant comme point de départ $v$. Comme $V_{1}$ et $V_{2}$
forment une partition de $V$, on remarque qu'aucun état ne peut être injouable:
il est toujours possible de jouer.

La dernière condition reflète le fait qu'on peut toujours sortir d'un état.
Cette notion n'est pas essentiel, mais simplifiera l'écriture des prochaines
définitions, des propositions s'y rapportant et leurs preuves. Le lecteur est
invité à omettre cette condition et à retravailler ce chapitre sans cette
dernière.

Les chemins du graphe $(V, E)$ sont appelés \textbf{les parties de $G$}.

\begin{definition} [Jeu combinatoire à somme nulle]
	Soit $G = ( (V, E), V_{1}, V_{2}, \Omega_{1}, \Omega_{2})$ un jeu
	combinatoire.

	On dit que $G$ est \textbf{à somme nulle} si $\Omega_{1} =
	\comp{\Omega_{2}}$.
\end{definition}

\begin{definition} [Partie maximale]
	Soit $G = ( (V, E), V_{1}, V_{2}, \Omega_{1}, \Omega_{2})$ un jeu
	combinatoire.

	Une partie $P$ de $G$ est dite \textbf{maximale} si celle-ci est infinie, ie
	si $P \in \sequencesIn{V}$.
\end{definition}

\begin{definition} [Partie gagnante]
	Soit $G = ( (V, E), V_{1}, V_{2}, \Omega_{1}, \Omega_{2})$ un jeu
	combinatoire.

	On dit qu'une partie $P$ maximale est \textbf{gagnante pour $J_{1}$ (resp.
	pour $J_{2}$)} si $P \in \Omega_{1}$ (resp. $P \in \Omega_{2}$).
\end{definition}

\begin{definition} [Partie perdante]
	Soit $G = ( (V, E), V_{1}, V_{2}, \Omega_{1}, \Omega_{2})$ un jeu
	combinatoire à somme nulle.

	Une partie $P$ maximale est \textbf{perdante pour $J_{2}$ (resp. perdante
	pour $J_{1}$)} si $P$ est gagnante pour $J_{1}$ (resp. gagnante pour
	$J_{2}$).
\end{definition}

Maintenant qu'on a défini les parties d'un jeu, il nous faut définir le terme de
stratégie.

\begin{definition} [Stratégie]
	Soit $G = ( (V, E), V_{1}, V_{2}, \Omega_{1}, \Omega_{2})$ un jeu
	combinatoire.

	\textbf{Une stratégie pour $J_{1}$} est une fonction
	\begin{equation}
		\lambda : V^{*}V_{1} \rightarrow V
	\end{equation}
	tel que pour tout $(v_{0}, \dots, v_{n}, v_{n + 1}) \in V^{*}V_{1}$, on a
	\begin{equation}
		(v_{n}, v_{n + 1}) \in E
	\end{equation}

	On définit de la même manière 'une stratégie pour $J_{2}$' en remplaçant
	$V_{1}$ par $V_{2}$.
\end{definition}

% Interprétation d'une stratégie.

\begin{definition}
	Soient $G = ( (V, E), V_{1}, V_{2}, \Omega_{1}, \Omega_{2})$ un jeu
	combinatoire, $P = \GSsequence{v}{i}{I}$ une partie maximale de $G$ et
	$\lambda_{1}$ une stratégie pour $J_{1}$.

	On dit que \textbf{$P$ est jouée en accord avec $\lambda_{1}$ par $J_{1}$}
	si pour tout $i \in I$
	\begin{equation}
		v_{i} \in V_{1} \implies \lambda_{1}( (v_{0}, \dots, v_{i}) ) = v_{i + 1}
	\end{equation}
	En d'autres termes, le joueur $1$ doit jouer exactement les états de la
	partie $P$.

	On définit de la même manière 'une partie est jouée en accord avec
	$\lambda_{2}$ par $J_{2}$' en remplaçant $V_{1}$ par $V_{2}$ et $\lambda_{1}$
	par $\lambda_{2}$.
\end{definition}

\begin{definition}
	Soient $G = ( (V, E), V_{1}, V_{2}, \Omega_{1}, \Omega_{2})$ un jeu
	combinatoire, $\lambda$ une stratégie pour $J_{1}$ et $v \in V$ un
	état.

	On définit
	\begin{equation}
		Outcome(\lambda, v)
	\end{equation}
	comme l'ensemble des parties maximales commençant en $v$ jouées en accord
	avec $\lambda$ par $J_{1}$.
\end{definition}
\end{document}


